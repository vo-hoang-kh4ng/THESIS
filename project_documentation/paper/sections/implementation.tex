\section{Implementation}

\subsection{Code Structure}
The implementation follows a modular architecture with clear separation of concerns:

\subsubsection{Core Components}
\begin{itemize}
    \item \textbf{Agent System}: Implemented in \texttt{agents.py}
    \item \textbf{Task Pipeline}: Defined in \texttt{tasks.py}
    \item \textbf{Flow Management}: Handled in \texttt{flows.py}
    \item \textbf{Utility Functions}: Stored in \texttt{my\_utils.py}
\end{itemize}

\subsection{Agent Implementation}
The agent system is built using CrewAI's framework, with each agent having specific roles and tools:

\subsubsection{Agent Configuration}
\begin{lstlisting}[language=Python]
class TaskConfig:
    def __init__(self, 
                 keywords: List[str] = None,
                 time_window: str = "24h",
                 crisis_threshold: float = 0.7):
        self.keywords = keywords or ["brand", "product", "service"]
        self.time_window = time_window
        self.crisis_threshold = crisis_threshold
\end{lstlisting}

\subsubsection{Agent Creation}
\begin{lstlisting}[language=Python]
def create_research_agent(config: TaskConfig) -> Agent:
    return Agent(
        role="Social Media Researcher",
        goal="Gather comprehensive data about brand presence",
        backstory="Expert in social media research and data collection",
        tools=[twitter_tool, web_crawler, social_media_search]
    )
\end{lstlisting}

\subsection{Task Implementation}
The task pipeline is implemented through a series of interconnected tasks:

\subsubsection{Research Task}
\begin{lstlisting}[language=Python]
def create_research_task(config: TaskConfig) -> Task:
    return Task(
        description="""
        Conduct systematic investigation of brand's online presence:
        1. Social Media Content
        2. News Coverage
        3. Market Intelligence
        4. Brand Performance
        """,
        expected_output="""
        Comprehensive research report with:
        1. Key events and trends
        2. Raw social media data
        3. Historical sentiment data
        4. Competitor analysis
        """
    )
\end{lstlisting}

\subsubsection{Monitoring Task}
\begin{lstlisting}[language=Python]
def create_monitoring_task(config: TaskConfig) -> Task:
    return Task(
        description="""
        Monitor social media metrics:
        1. Volume metrics
        2. Engagement metrics
        3. Reach metrics
        4. Performance metrics
        """,
        expected_output="""
        Detailed monitoring report with:
        1. Volume analysis
        2. Engagement insights
        3. Performance benchmarks
        4. Competitive analysis
        """
    )
\end{lstlisting}

\subsection{Crisis Detection}
The crisis detection system implements sophisticated algorithms for identifying potential crises:

\subsubsection{Crisis Signal Detection}
\begin{lstlisting}[language=Python]
def detect_crisis_signals(sentiment_result: Dict) -> bool:
    # Extract key metrics
    sentiment_trend = sentiment_result.get("sentiment_trend", "")
    engagement_spike = sentiment_result.get("engagement_spike", False)
    viral_content = sentiment_result.get("viral_content", False)
    
    # Define crisis conditions
    crisis_conditions = [
        "High negative sentiment" in sentiment_trend,
        "Urgent issues mentioned" in sentiment_trend,
        "Viral content detected" in sentiment_trend,
        engagement_spike,
        viral_content
    ]
    
    return any(crisis_conditions)
\end{lstlisting}

\subsection{Data Processing}
The system implements efficient data processing techniques:

\subsubsection{Sentiment Analysis}
\begin{lstlisting}[language=Python]
def analyze_sentiment(text: str) -> Dict:
    return {
        "basic_sentiment": classify_sentiment(text),
        "emotions": detect_emotions(text),
        "aspects": analyze_aspects(text),
        "themes": identify_themes(text)
    }
\end{lstlisting}

\subsubsection{Data Collection}
\begin{lstlisting}[language=Python]
def collect_social_media_data(platform: str, query: str) -> List[Dict]:
    tools = {
        "twitter": twitter_tool,
        "web": web_crawler,
        "social": social_media_search
    }
    return tools[platform].search(query)
\end{lstlisting}

\subsection{Error Handling}
The implementation includes robust error handling:

\subsubsection{Exception Management}
\begin{lstlisting}[language=Python]
def safe_execute(func: Callable, *args, **kwargs) -> Any:
    try:
        return func(*args, **kwargs)
    except Exception as e:
        logger.error(f"Error in {func.__name__}: {str(e)}")
        return None
\end{lstlisting}

\subsection{Performance Optimization}
The system implements several optimization techniques:

\subsubsection{Caching}
\begin{lstlisting}[language=Python]
@lru_cache(maxsize=1000)
def get_cached_data(key: str) -> Dict:
    return fetch_data(key)
\end{lstlisting}

\subsubsection{Parallel Processing}
\begin{lstlisting}[language=Python]
def process_data_parallel(data: List[Dict]) -> List[Dict]:
    with ThreadPoolExecutor() as executor:
        return list(executor.map(process_item, data))
\end{lstlisting}

This implementation provides a robust and efficient system for social media monitoring and crisis detection, with careful attention to performance, error handling, and maintainability. 